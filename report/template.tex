\documentclass[letterpaper,10pt,titlepage]{article}

\usepackage{graphicx}
\usepackage{amssymb}
\usepackage{amsmath}
\usepackage{amsthm}

\usepackage{alltt}
\usepackage{float}
\usepackage{color}
\usepackage{url}

\usepackage{pstricks, pst-node}

\usepackage{geometry}
\geometry{textheight=8.5in, textwidth=6in}

%random comment

\newcommand{\cred}[1]{{\color{red}#1}}
\newcommand{\cblue}[1]{{\color{blue}#1}}

\usepackage{hyperref}
\usepackage{geometry}

\usepackage{listings}
\lstset{
  language=C++,
  aboveskip=3mm,
  belowskip=3mm,
  showstringspaces=false,
  columns=flexible,
  basicstyle={\small\ttfamily},
  numbers=none,
  numberstyle=\tiny\color{gray},
  keywordstyle=\color{blue},
  commentstyle=\color{dkgreen},
  stringstyle=\color{mauve},
  breaklines=true,
  breakatwhitespace=true,
  tabsize=3
}
\setlength{\parindent}{0cm}
\setlength{\parskip}{0.8em}

\def\name{Lane Breneman,Joshua Deare, Hari Caushik}
\title{Stopify - Real time road sign detection}
\date{June 1st, 2015}
\author{Lane Breneman,Joshua Deare, Hari Caushik}

\begin{document}
\maketitle

\section*{Introduction}
%Who requested it?
%Why was it requested?
%What is its importance?
%Who was your client? 
%Who are the members of your team?
%What were their roles?
%What was the role of the client(s)? (I.e., did they supervise only, or did they participate in doing development) 
Our clients, ON Semiconductor requested that we research and implement computer
vision algorithms to be able to detect street signs from video or still images
intended to be captured from a low cost camera inside an automobile. The 
algorithms were required to run on low cost development boards like the 
Beaglebone Black and the NVIDIA Jetson TK1. ON Semiconductor recently acquired
Aptina, the leading producer of CMOS image sensors and hope to enter into the
Automotive Safety industry sometime in the future. They intended this to be a 
research project in which we can research, develop and test a variety of stop 
sign detection algorithms and analyze them to determine which algorithms seem
most promising, which development boards are most promising and provide a base 
for future senior design projects to contribute to.

Our clients at ON Semiconductor were Matthew Zochert, Carl Price and Don Reid.
The members of our team were Joshua Deare, Lane Breneman and Hari Caushik. 
Throughout the year, we each contributed to the creation of the algorithm 
development framework, testing framework, capturing of images and video for 
the test dataset and presenting our work. Our clients tracked our progress,
provided suggestions for our algorithms, use of hardware, and the presentation
of our work and even provided us with starter code early on to ramp up on 
programming with the OpenCV C++ library. 

\subsection*{Main Project Changes}

\subsection*{Weekly Blog Posts}

\subsection*{Project Documentation}
% How does your project work?
% - What is its structure?
% - What is its Theory of Operation?
% - Add Block and flow diagrams
% How does one install your softare, if any?
% How does one run it?
% Are there any special hardware, OS, or runtime requirements to run your
%  software?
% Any user guides, API documentation, etc.
Our project consists of four main components: the development framework, 
algorithms, testing framework and test dataset. The development framework
provides a well defined interface to develop the computer vision algorithms and
runs the algorithms on provided images and displays the output. The algorithm 
used by the development framework is linked based on rules provided by a 
Makefile. The algorithms process the images and determine whether they contain
a stop sign or not. The testing framework is used to test the effectiveness of
each of our algorithms based on a number of evaluation metrics. It essentially
invokes the development framework and the algorithm linked to it on each image
in the test dataset.

\subsubsection*{Development Framework}
\paragraph*{Usage}
The development framework, written in
framework\_images/framework\_cpp/framework.cpp from the root project directory,
is a C++ program that takes a command line argument, the path of the image to
process, and returns a 1 if the algorithm determined that the image contained a
stop sign or a 0 if not. An optional third argument can be provided to display
the image being classified. For example, to classify 
framework\_images/framework\_cpp/pos\_sample.jpg, an image containing a stop
sign, the proper usage is
\begin{lstlisting}
./framework pos_sample.jpg
\end{lstlisting}
to simply return the classification result without displaying it. To display
the image, this would be
\begin{lstlisting}
./framework pos_sample.jpg 1
\end{lstlisting}
although the actual value of the third argument does not matter, just its 
presence.
\paragraph*{Implementation Details}
The development framework includes the
framework\_images/framework\_cpp/algorithm.h header file that provides the 
interface the algorithms must be written to. This is the processImage function,
with the signature
\begin{lstlisting}
int processImage(Mat &img);
\end{lstlisting}
The framework loads the test image from the path specified in the command line 
argument onto memory with the OpenCV imread funtion as follows:
\begin{lstlisting}
Mat img = imread(argv[1], CV_LOAD_IMAGE_COLOR);
\end{lstlisting}
The second argument specifies the loaded image to be a 3 channel color image
and the output is stored in an OpenCV Mat object, a wrapper around a 
multi-channel 2-dimensional image array.

\subsubsection*{Algorithms}

\subsubsection*{Testing Framework}

\subsubsection*{Test Dataset}

\subsection*{Resources}
% What web sites were helpful (Listed in order of helpfulness)
% What, if any, reference books really helped?
% Were there any people on campus that were really helpful?
The following resources were particularly helpful:

\subsection*{What We Learned}
% What technical information did you learn?
% What non-technical information did you learn?
% What have you learned about project work?
% What have you learned about project management?
% What have you learned about working in teams?
% If you could do it all over, what would you do differently?
\subsubsection*{Hari Caushik}
By contributing to this project, I learned a fair bit of technical information. 
I improved my C++ programming skills and gained a familiarity with the OpenCV
library. In addition, I learned about the major concepts in computer vision
algorithms as well as image processing techniques. Throughout this project, I
also learned about software design practices and made practical use of
algorithmic complexity analysis.

A lot of non-technical information was also learned by doing this project. This
project required a lot of documentation at the start while we were 
determining the specifications, during the implementation when we documented 
the pseudocode and algorithmic complexity of the algorithms as well as our 
final report. My technical writing skills have improved as a result. We also 
presented our project, particularly towards the end of the year at the 
Engineering Expo to the general public, at ON Semiconductor to ON employees and
our final presentation video. 

Doing a nine month project for an actual client in industry has taught me most
of all that the main challenge is carefully defining an initially very 
loosely defined problem. Decisions that we make early on can have a very 
significant effect on the way our project may turn out. I have also learned 
that projects in the real world are very malleable. While the high level 
decisions we make early on guide our project in certain directions, specific 
requirements frequently change as we gain more knowledge through 
implementation. 

This project allowed each of use to take a project management role at different
times throughout the year. I learned most of all that establishing clear 
communication among all team members is essential to moving the project along 
smoothly. It is much easier to perform when each of us knows what is expected
of us and have clearly defined responsibilities. When this breaks down, there 
tends to be a lot of confusion and consequently inaction. 

\subsubsection*{Lane Breneman}

%What technical information did you learn?
Before this project I had used OpenCV for object recognition. That said I had
no experience working with the C++ interface and very little C++ experience in
general. Through this project I learned a lot about linking libraries, creating
make files and using classes in C++. On a different note I also learned a lot
about SVMs and machine learning. I had attempted to use these tools but had
never made much progress, but after this project I feel I have the skills to
employ them in future projects.

%What non-technical information did you learn?
I learned about taking notes and extracting the important information from 
research papers in order to reimplement their algorithms. I also learned 
how important communication between team members is. This largely came from
the fact the two of our basic algorithms turned out to be very similar and 
two of our advanced algorithms are also kind of similar. This limited the
variety of algorithms we explored, thus lowering the quality of the project.

%What have you learned about project work?
Our project integrated really nicely together, a large part of this I 
attribute to our good design. We found that we almost never needed to
fight the framework in order to get our code to run, which is nice. 
Even when last minute we need to make a way to export videos it took 
about 5 minutes to extend the framework. This reinforced the most 
import fact that I knew about project work, that a good design is 
important.

%What have you learned about project management?
Project management was a very mixed bag for our project. In the 
beginning we had very good project management, planning everything,
dividing work and communicating often. This lead to the best parts
of out project namely the framework and testing framework. Our later work
were less thought out and more divided and I feel that it can be seen in
quality of the work, as our algorithms could have been better.

%What have you learned about working in teams?
Communication is key. We had a really great group and I enjoyed working on the
project, but for most of the term the project became mostly individual. This lead
to two of our algorithms being pretty much the same. If we had communicated in 
more detail we could have seen this sooner and adapted.


%If you could do it all over, what would you do differently? 
If I could make one change I would have done 1 advanced algorithm each that we worked on over the year.
By advanced I mean more complicated than even the HOG or SURF algorithms. I feel
the project would of had more to show and been more rewarding if we had.

\subsubsection*{Joshua Deare}

\subsection*{Appendix 1: Essential Code Listings}

\subsection*{Appendix 2: Other Material}

\end{document}
